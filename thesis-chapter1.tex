\chapter{РАЗРАБОТКА ПРОГРАММНОГО ИНТЕРФЕЙСА}

Сначала в данной главе будет произведен поверхностный обзор интерфейса прикладного программирования робота. Подробно рассматриваться данный компонент не будет в связи с тем, что все управление роботом в дальнейшем будет производиться через методы классов, предоставляемых симулятором Webots. Взаимодействие с интерфейсом DARwIn Framework будет происходить только при компиляции модуля для интерпретации непоследственно на самом роботе. Это связано с тем, что вызовы методов классов Webots для управления моторами на физической модели робота имеют существенные задержки во времени, которые не позволяют роботу совершать плавные и непрерывные движения.

Так же кратко будет рассмотрен интерфейс управления приводами одного из самых распространенных гуманоидных роботов Nao. Этот анализ будет произведен для получения представления о имеющихся аналогах систем управления: их плюсах и недостатках.

Далее, исходя из ранее полученных знаний, будет разработан интерфейс класса для языка программирования C++. Другие популярные языки высокого, такие как Java и Python, были исключены из-за их низкой производительности. Для поддержки данных языков в реализованного модуля управления можно воспользоваться инструментом связывания программ и библиотек, написанных на языке C и C++, с интерпретируемыми (Tcl, Perl, Python, Ruby, PHP) или компилируемыми (Java, C\#, Scheme, OCaml) языками - SWIG.


