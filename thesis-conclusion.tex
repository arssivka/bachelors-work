\chapter*{ЗАКЛЮЧЕНИЕ}
\addcontentsline{toc}{chapter}{ЗАКЛЮЧЕНИЕ}

В ходе данной работы был проведен анализ программного обеспечения для программирования и разработки робототехнических систем и была реализована библиотека для управления механикой гуманоидного антропоморфного робота Darwin-op.

Основной пакет программирования Darwin Framewrok имеет поддержку управления роботом на низком уровне. Так же в реализации алгоритма походки были обнаружены ошибки. Для Darwin-OP отсутствует гибкая система управления механикой робота. Для этого в ходе работы была разработана система, позволяющая управлять механикой робота в декартовой системе координат. Так в ходе работы было выяснено, что конструкция рук робота Darwin не позволяет воспроизводить управление в декартовой системе координат.

Анализ симулятора Webots показал, что симуляционная среда корректно поддерживает все основные функции симуляции робота Darwin-Op кроме воспроизведения звука. Набор классов, предоставленных симулятором реализуют удобный интерфейс для взаимодействия с механикой робота и упрощает процесс инициализации всей системы. Программный код разрабатывался с учетом поддержки симуляции и для запуска на реальном роботе. На основе проанализированных классов был реализован код, который позволял интерпретировать управление роботом в симуляционной среде. Так же опытным путем было исследовано, что для управления группой моторов на реальном роботе не следует использовать классы Motors из-за большой задержки при отправлении данных на сервоприводы.

При разработке интерфейса был проанализирована структура класса управления механикой гуманоидного робота Nao. Данный класс содержит большой набор методов управления, часть которых следует разместить в отдельных классах из-за того, что они не являются низкоуровневыми механизмами взаимодействия с роботом. Интерфейс методов для работы с отдельными моторами и частями тела в декартовой системе координат взят для организации поддержки разных робототехнических и методов управления роботом при разработке собственной библиотеки управления.

По результатам анализа были выдвинуты требования, на основе которых разрабатывался интерфейс системы управления. Система была разбита на две группы: класс, который выполняет управление роботом с помощью описания параметров отдельных частей робота и класс, который описывает механизм сложного движения. Между собой эти классы являются независимыми. Это позволит разработчикам изменять низкоуровневую систему управления без изменения кода разработанных движений. Так же при разработке были учтены особенности архитектур разных роботов и возможность переноса библиотеки на разные платформы. 

Для управления механикой робота была решена задача обратной кинематики. Алгоритм позволяет управлять положением и ориентацией туловища и ног в декартовой системе координат. Задача была решена с помощью геометрического подхода. Такой подход позволил сократить количество вызываемых тригонометрических функций, в сравнении с черешнием уравнений через матрицы вращения и переноса, и немного увеличить производительность. 