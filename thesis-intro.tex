\chapter*{ВВЕДЕНИЕ}
\addcontentsline{toc}{chapter}{ВВЕДЕНИЕ}

В данной работе основной целью является разработка программного компонента управления приводами гуманоидного антропоморфного робота DARwIn-OP (Dynamic Anthropomorphic Robot with Intelligence–Open Platform), разработанного корейской компанией  Robotis, с поддержкой симуляции данного компонента в среде Webots.

В ходе работы были проанализированны имеющиеся программные компоненты управления в фреймворке робота и в симуляционной среде Webots. Так же для разработки интерфейса взаимодействия с роботом будет рассмотрен компонент управления механикой робота из комплекта средств разработки (SDK) гуманоидного робота Nao, выпущенного компанией Aldebaran Robotics. После детального анализа в данной работе был разработан алгоритм, интерфейс и программная реализация мезанизма управления сервоприводами робота DARwIn-OP.

На сегодняшний день в робототехнике активно ведется разработка различных гуманоидных роботов с различными конструктивными характеристиками и параметрами, но общая модель гуманоидного робота остается неизменной. Исходя из этого при разработке интерфейса взаимодействия нужно поставить акцент на возможности реализации алгоритмов управления и на других роботах.

Данный программный продукт будет распространяться под лицензией с открытым исходным кодом GNU General Public License v3. Для публикации исходного когда и дальнейшего развития продукта будет создан git репозиторий на веб-сервисе GitHub. Данный веб-сервис является самым крупным хостингом для IT-проектов и их совместной разработки. Так же это позволит всем заинтересованным лицам так же принимать участие в разработке данной программной библиотеки.

Так же стоит заметить, что система управления роботом разрабатывается с упором возможности программирования движений робота для игры в футбол. Так же у DARwIn-OP довольно ограниченные возможности программирования движения рук из-за конструктивных особенностей робота и именно из-за этого при разработки системы управления будет отсутствовать реализация управлением позициями кистей рук.